% Created 2018-06-17 Sun 21:47
\documentclass[14pt, c]{beamer}
\usepackage[utf8]{inputenc}
\usepackage[T1]{fontenc}
\usepackage{fixltx2e}
\usepackage{graphicx}
\usepackage{longtable}
\usepackage{float}
\usepackage{wrapfig}
\usepackage{rotating}
\usepackage[normalem]{ulem}
\usepackage{amsmath}
\usepackage{textcomp}
\usepackage{marvosym}
\usepackage{wasysym}
\usepackage{amssymb}
\usepackage{hyperref}
\tolerance=1000
\usepackage{lmodern}
\usepackage{soul}
\beamerdefaultoverlayspecification{<*>}
\usetheme{Warsaw}
\usecolortheme{seahorse}
\useinnertheme{circles}
\useoutertheme{shadow}
\setcounter{secnumdepth}{3}
\author{\emph{Group 9 ICT}}
\date{\today}
\title{RDBMS Application in Business Management}
\hypersetup{
  pdfkeywords={},
  pdfsubject={},
  pdfcreator={Emacs 25.3.1 (Org mode 8.2.10)}}
\begin{document}

\maketitle
\setbeamercolor{block title}{fg=black,bg=white}
\setbeamertemplate{blocks}[rounded][shadow=false]

\section{}
\label{sec-1}
\begin{frame}[label=sec-1-1]{Why Use a Relational Database}
\begin{block}{}
\begin{itemize}
\item Customer Management\pause
\item Inventory Tracking\pause
\item Personnel Database\pause
\item Analysis
\end{itemize}
\end{block}
\end{frame}

\begin{frame}[label=sec-1-2]{Our Project}
\begin{block}{}
\begin{enumerate}
\item The Organization
\begin{itemize}
\item \alert{Authentic Mexican Food (AMF)}: An \emph{imaginary} restaurant based in New York City\pause
\end{itemize}
\item The Database
\begin{itemize}
\item \alert{amf.sql}: Contains the SQL queries to create tables for the database\pause
\end{itemize}
\item The Software
\begin{itemize}
\item \alert{Microsoft SQL Server}: A database system that has been in use since \emph{1989}
\end{itemize}
\end{enumerate}
\end{block}
\end{frame}

\begin{frame}[label=sec-1-3]{Schema}
\pause
\begin{block}{}
\begin{enumerate}
\item Back-end
\begin{itemize}
\item Employees, Storage, Suppliers\pause
\end{itemize}
\item Middle-end
\begin{itemize}
\item Orders, Deliveries, Takeaways\pause
\end{itemize}
\item Front-end
\begin{itemize}
\item Foods, Tables, Customers
\end{itemize}
\end{enumerate}
\end{block}
\end{frame}

\begin{frame}[label=sec-1-4]{Contents and Purposes of Each Table}
\pause
\begin{block}{}
\small{}
\alert{Employees}: Name, Age, Address, Phone, Role, Salary\\
\alert{Storage}: Product, Quantity, Date\\
\alert{Suppliers}: Name, Address, Phone, Product\pause
\end{block}
\begin{block}{}
\small{}
\alert{Orders}: Food, Table, Customer, Employee\\
\alert{Deliveries}: Food, Adress, Date, Customer, Employee\\
\alert{Takeaways}: Food, Customer, Employee\pause
\end{block}
\begin{block}{}
\small{}
\alert{Foods}: Name, Category, Price\\
\alert{Tables}: Seats, FloorNumber\\
\alert{Customers}: Name, Gender, Age, Adress, Phone
\end{block}
\end{frame}

\begin{frame}[label=sec-1-5]{How To Use The Database}
\begin{block}{}
\begin{enumerate}
\item \small{}When the customer placed an order: Insert to \alert{Orders} with \emph{FoodID}, \emph{TableID}, \emph{CustomerID}, \emph{EmployeeID}
\item \small{}When the customer called the restaurant and ordered shipping: Insert to \alert{Deliveries}, same as above and added \emph{DeliveryAddress}, \emph{DeliveryDate}
\item \small{}When the customer ordered food to bring away: Insert to \alert{Takeaways}, same as above but without \emph{TableID}
\item \small{}When products from a supplier come: Update \alert{Storage} with new \emph{Quantity} and \emph{Date}
\end{enumerate}
\end{block}
\end{frame}

\begin{frame}[label=sec-1-6]{}
\begin{block}{}
\begin{center}
\huge{}Examples
\end{center}
\end{block}
\end{frame}
% Emacs 25.3.1 (Org mode 8.2.10)
\end{document}
